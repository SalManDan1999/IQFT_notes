%%
%% Author: Dario Chinelli
%% begin 2022-10-16
%% last mod 2022-12-24
%%


%%  NON ESEGUIRE QUESTO FILE !


% Preamble
\documentclass[class=article]{standalone}

% Packages
\usepackage[subpreambles=true]{standalone}
\usepackage{import}
\usepackage{graphicx}
\usepackage{amsmath}


% Document
\begin{document}


\vspace{30mm} % \newpage
\section{22 settembre}

\vspace{30mm} % \newpage
\section{23 settembre}

\vspace{30mm} % \newpage
\section{27 settembre}


\newpage
\section{4 ottobre}
\paragraph{The rod}
Given a 1-dimensional rod composed by N-particles, linked each others with a "spring", the hamiltonian density is

\begin{equation*}
\begin{split}
\mathcal{H} = \frac{1}{2} \sum_{n=1}^{N} \Big[  P_n^2 + \Omega^2 (q_n - q_{n+1})^2 + \Omega_0^2 q_n^2 \Big]
\end{split}
\end{equation*}

where the last term $\Omega_0^2 q_n^2$ is relative to the equilibrium position of the n-particle.
The \emph{periodic boundaries conditions} to $N \to \infty$ and $a \to 0$.

On the other side we can write the Newtonian equation as 

\begin{equation*}
\begin{split}
& H = \frac{1}{2} \int_{0}^{L} dx \Big[ p^2(x) + v^2 \Big( \frac{\partial q(x)}{\partial x} \Big) \Big] \\
% & \mbox{where} \\
&  p(x) = \dot q(x) \\
&  \ddot q(x) = v^2 \frac{\partial^2 q(x)}{\partial x^2}
\end{split}
\end{equation*}

the solution inside the boundaries is
\begin{equation*}
\ddot q_n = \Omega^2 \Big( q_{n+1} + q_{n-1} -2 q_n \Big)
\end{equation*}


\paragraph{Normal modes} or normal frequencies
\begin{equation*}
\begin{split}
& q_n = \sum_j e^{i j n} \frac{Q_j}{\sqrt{N}} \\
& q(x) =  \frac{1}{\sqrt{a} } \sum_n e^{ \frac{2 \pi l}{N a} (n a) } \frac{Q_j}{\sqrt{N}}  = \frac{1}{\sqrt{a} }  \sum_k e^{i k x} \frac{Q_k}{\sqrt{N}} \\
& \quad\quad k = \frac{2 \pi l}{L} \\
& \Rightarrow\quad q(x) =  \sum_k e^{i k x} \frac{Q_k}{\sqrt{N a}} =  \sum_k e^{i k x} \frac{Q_k}{\sqrt{L}}
\end{split}
\end{equation*}

Considering now the Newtonian equation, $p^2(x) = \dot q^2(x) $, 
$ \sum_{n=1}^N e^{i n (j-j')} = \delta_{j, j'}$ where $j = \frac{2\pi l}{N}$, 
we can move from the sum to the integral using the following relation 
$ \sum_{n=1}^N \to \frac{1}{a} \int_{0}^{L} dx$ and this leads to
$\int_{0}^{L} dx e^{i (k-k') x} = L \delta_{k, k'}$ .

Somehow we may land on this following expression:
\begin{equation*}
\frac{1}{L} \sum_{k, k'} L \, \delta_{k, k'} Q_k \dot Q_{k'} = \sum_k Q_k \dot Q_{k} =   \sum_k | \dot Q_{k} |^2
\end{equation*}
To finally get a \emph{total classical description}: a discrete sum on a numerable set, as follow

\begin{equation*}
H = \frac{1}{2}  \sum_k  | \dot Q_{k} |^2 + k^2 v^2  | Q_{k} |^2
\end{equation*}

As before, notice that the sum $ \sum_{n=1}^N $ for $L \to \infty$ became $\frac{L}{2 \pi} \int dk$ and it admits waves.
Extending this to 3-dimensional space, it became
\begin{equation*}
 \sum_{\vec k} (\ldots) \quad (\mbox{when }L \to \infty) \quad \frac{V}{(2\pi)^3} \int d^3 k 
\end{equation*}

\paragraph{Quantum system:}  let's consider now a quantum system, a quantum description. \\
\emph{Postulate} the followings:

\begin{equation*}
\begin{split}
& \Big[ q_l , p_n \Big] = i \, \delta_{l\,n} \\
& \Big[ q_l , q_n \Big] = 0 \\
& \Big[ p_l , p_n \Big] = 0 \\
\end{split}\quad\quad\quad
\begin{split}
& \Big[ Q_l , P_n \Big] = i \, \delta_{ln} \\
& \Big[ Q_l , Q_n \Big] = 0 \\
& \Big[ P_l , P_n \Big] = 0
\end{split}\quad\quad\quad
\begin{split}
& \mbox{Where natural units are applied:} \\
& \quad\quad h = 1 \\
& \quad\quad c = 1
\end{split}
\end{equation*}

\begin{equation*}
\begin{split}
& \Rightarrow\quad\quad q_n^{\dagger} = q_n  \quad , \quad Q_{-j} = Q_{j}^{\dagger} \quad , \quad  P_{-j} = P_{j}^{\dagger} \\
\mbox{e.g.} & \quad q_n^{\dagger} = \Big( \sum_n e^{i n j} \frac{Q_j}{\sqrt{N}} \Big)^{\dagger} =  \sum_j e^{ - i n j} \frac{Q_j^{\dagger}}{\sqrt{N}} = q_n
\end{split}
\end{equation*}

From the hamiltonian
\begin{equation*}
\mathcal{H} = \frac{1}{2} \sum_j \Big[ P_j\,P_j^{\dagger} + \omega_j^2 \, Q_j\,Q_j^{\dagger}  \Big]
\end{equation*}
and given the following operators, we find $Q_j$ and $P_j$:
\begin{equation*}
\begin{split}
& a_j = \frac{1}{\sqrt{2\omega_j}} \Big( \omega_j Q_j + i\,P_j^{\dagger} \Big) \\
& a_j = \frac{1}{\sqrt{2\omega_j}} \Big( \omega_j Q_j^{\dagger} - i\,P_j \Big) 
\end{split}\quad\quad\quad\Rightarrow\quad\quad
\begin{split}
& Q_j = \frac{1}{\sqrt{2\omega_j}} \Big( a_j + a_{-j}^{\dagger} \Big) \\
& P_j = -i \Big( \frac{\omega_j}{2} \Big)^{\frac{1}{2}} \Big( a_{-j} - a_j^{\dagger} \Big)
\end{split}\quad\quad\quad\quad \mbox{and} \quad\quad
\begin{split}
& \mbox{keep in mind} \\
& \Big[ a_j , a_{j'} \Big] = \delta_{j \, j'}
\end{split}
\end{equation*}
\begin{equation*}
\begin{split}
& Q_j Q_j^{\dagger} = \frac{1}{2 \omega_j} \Big( a_j a_j^{\dagger} + a_j a_{-j} + a_{-j}^{\dagger} a_j^{\dagger} + a_{-j}^{\dagger} a_{-j} \Big) \\
& P_j P_j^{\dagger} =  \Big( \frac{\omega_j}{2} \Big)^{\frac{1}{4}} \Big( a_{-j} a_{-j}^{\dagger} - a_{-j} a_j - a_j^{\dagger}a_{-j}^{\dagger} + a_j^{\dagger} a_j \Big)
\end{split}
\end{equation*}
With these lasts results we may write the $\mathcal{H}$ as
\begin{equation*}
\begin{split}
\mathcal{H} & = \frac{1}{2} \sum_j \Big[ P_j\,P_j^{\dagger} + \omega_j^2 \, Q_j\,Q_j^{\dagger}  \Big]
 =  \frac{1}{2} \sum_j \omega_j \Big( a_j a_j^{\dagger} + a_j^{\dagger} a_j \Big) \\
& =  \frac{1}{2} \sum_j  \omega_j \Big( 2 \, a_j^{\dagger} a_j + 1 \Big)
= \sum_j  \omega_j \Big( a_j^{\dagger} a_j + \frac{1}{2} \Big)
\end{split}
\end{equation*}

\paragraph{Phonons description} Phonons are bosons, they're used to describe the quantum problem of the rod. Phonons are like photons but in the world of sound instead of light.
A n-particles system is defined with 
\begin{equation*}
\ket{n_1, n_2, n_3, \ldots } = (a_1^{\dagger})^{n_1}  (a_2^{\dagger})^{n_2}  (a_3^{\dagger})^{n_3} \ldots \ket{0}
\end{equation*}
and for the 1-d oscillator, with energy $E_n$, is as follows
\begin{equation*}
\begin{split}
& \ket{n} =  (a^{\dagger})^{n} \ket{0} \\
& E_n = \hbar \omega (n + \frac{1}{2}) \stackrel{nu}{=} \omega (n + \frac{1}{2}) 
\end{split}
\end{equation*}
For the phonons is easy to \emph{understand} which is the medium that make the transmission possible, but what about light?
For the light, photons, the medium may also be the \emph{vacuum}.

\begin{center}
Filosofeggiamo un po' ora:
\fbox{\begin{minipage}{42em}
\begin{center}
\large \emph{Particles are the excitation of the field} \\
\normalsize If you don't touch the piano it stays quiet, but if you play it it makes music ... song's particles.
\large \emph{The field is permanent.} \\
\normalsize \emph{Particles are not fixed, they live and die.} \\
You cannot touch or see the field that you're studying, but you can see/detect the particle that pop out from the field. \\
\large Fields are NOT real but mathematical description of the world.
\end{center}
\end{minipage}}
\end{center}

When you measure an energy it's always relative to an offset, a ground-state.
Because you want the \emph{vacuum} to be Lorentz invariant.
\begin{equation*}
\quad\Rightarrow\quad 
\Big( \mathcal{H} - E_0 \Big) \ket{0} = 0
\end{equation*}

Given a general operator $\Theta (t)$ and its derivate $\dot \Theta = i \Big[ \mathcal{H}, \Theta(t) \Big]$ so that: 

\begin{equation*}
\begin{split}
\dot a(t) = i  \Big[ \mathcal{H}, a(t) \Big] = - i \, \omega a(t)
\end{split} 
\quad\quad \mbox{where}\quad
\begin{split}
& \Big[ a ,  \mathcal{H}\Big]  = \omega a \\
& \Big[ \mathcal{H} , a^{\dagger} \Big]  = \omega a^{\dagger}
\end{split}
\quad\quad\Rightarrow\quad
\begin{split}
a(t) = e^{-i \omega t} a(0)
\end{split}
\end{equation*}

(Finire lezione del 4 ottobre manca mezza pagina di esercizio - chiedere appunti)


\newpage
\section{6 ottobre}

\begin{equation*}
\begin{split}
\phi(x) & = \sum_{\vec k} \frac{1}{\sqrt{2 \omega_k V}} \Big( a_{\bar k} e^{i k x} +  a_{\bar k}^{\dagger} e^{-i k x} \Big) \\
& = \sum_{\vec k} \frac{i}{2 V} \Big[ e^{i\,\vec k (\vec x \cdot \vec y)} + e^{- i\,\vec k (\vec x \cdot \vec y)} \Big]
\end{split} \quad\quad\quad\mbox{considering} \quad
\begin{split}
& \Big[ a_k , a_{\bar k}^{\dagger}  \Big] = \delta_{\bar k, \bar k'} \\
& \Big[ \phi(\vec x, t), \dot\phi(\vec y, t) \Big] = i \, V \delta^3 (\vec x - \vec y)
\end{split}
\end{equation*}

\begin{equation*}
\begin{split}
&\mbox{Da capire che senso ha} \\
&\mbox{e contestualizzarlo} \quad\Rightarrow
\end{split}\quad\quad\quad\quad
\begin{split}
& k_{\mu} = (\vec k , i \omega_{\vec k}) \\
& x_{\mu} = (\vec x , i t) \\
& k_{\mu} x_{\mu} = \vec k \cdot \vec x = k_{\mu} k_{\nu} \delta_{\mu\,\nu} \\
& k_{\mu} x_{\mu} = \vec k \cdot \vec x - \omega_{\vec k} t
\end{split}
\end{equation*}


\emph{When things go to infinity} $\sum_{\vec k}  \rightarrow \frac{V}{(2\pi)^3} \int d^3 \vec k$ and remember that \say{\emph{if things doesn't work there will be some volume $V$ somewhere}}.
Creation and destruction operators are contained into the description of the field.
The energy levels' order are given from the term $ n \omega$ and you can forgot about the $\frac{1}{2}$.

\paragraph{Classical problem} Given the coordinates $q_i(t)$ time-dependents, where $i = 1, 2, 3, \ldots, 3N$, we can write the system of the $2^o$ order derivate as follow
\begin{equation*}
\begin{split}
& F_i = m \ddot q_i \\
& F_i = - \frac{d V}{d q_i}
\end{split}
\quad\quad \mbox{given the initial conditions}\quad
\begin{split}
& q_i(t_0) \\
& \dot q_i(t_0) 
\end{split}
\quad\quad \mbox{or given the boundary conditions}\quad
\begin{split}
& q_i(t_1) \,,\, q_i(t_2)   \\
& \dot q_i(t_1) \,,\, \dot q_i(t_2)  
\end{split}
\end{equation*}

\paragraph{Action Functional} The \emph{Action Functional $S$} 
\begin{equation*}
S = \int_{t_1}^{t_2} dt \, L \Big( q_i(t),\dot q_i(t) \Big)
\end{equation*}
is defined such that a variation on the trajectory leads to a variation on $S$.
So we can make a variables' transformation such that the new coordinates are the same as before plus a variational term
\begin{equation*}
\begin{split}
& q_i(t) \rightarrow q_i(t) + \delta q_i(t) \\
& \dot q_i(t) \rightarrow \dot q_i(t) + \delta \dot q_i(t) =  \dot q_i(t) + \frac{d}{dt} \delta q_i(t)
\end{split}
\end{equation*}
hence the action became

\begin{equation*}
\begin{split}
\delta S & = \int_{t_1}^{t_2} L \Big( q_i(t) + \delta q_i(t) ,  \dot q_i(t) + \frac{d}{dt} \delta q_i(t) \Big) dt -  \int_{t_1}^{t_2} dt \, L \Big( q_i(t),\dot q_i(t) \Big) dt = \\
& =  \int_{t_1}^{t_2} \, \Big(   \frac{\partial L}{\partial q_i} \delta q_i(t)    , \frac{\partial L}{\partial \dot q_i}  \frac{d}{dt} \delta q_i(t)    \Big)  dt
=  \int_{t_1}^{t_2} \Big(  \frac{\partial L}{\partial q_i}  - \frac{d}{dt}  \frac{\partial L}{\partial \dot q_i}   \Big) \delta q_i \, dt
\end{split}
\end{equation*}
in the last step we used the boundary condition at $t_1$ and $t_2$, so that $\delta q_i(t_1) = \delta q_i(t_2) = 0$.
The last step leads directly to the \emph{lagrangian equation}, that referred to the following generic $L$ in 3-D is:
\begin{equation*}
\begin{split}
& \frac{d}{dt} \Big( \frac{\partial L}{\partial \dot q_i} \Big) = \frac{\partial L}{\partial q_i} \\
& L = \frac{1}{2} m \dot\vec q^2 - V(\vec q)
\end{split}
\end{equation*}
this leads to the Newton equation for the \emph{free motion}:
\begin{equation*}
\begin{split}
& F = 0 \\ 
& m \ddot q_i = - \frac{\partial V}{\partial q_i }  = F
\end{split}\quad\Rightarrow\quad
\begin{split}
& \ddot \vec q = 0 \\
& \dot \vec q = \vec w \\
& \vec q = \vec w t + \vec r
\end{split}\quad\Rightarrow\quad
\begin{split}
& \vec q_1 = \vec w t_1 + \vec r \\
& \vec q_2 = \vec w t_2 + \vec r
\end{split}
\end{equation*}
so now we can find the Lagrangian depending on $q_1,q_2$ and $t_1,t_2$
\begin{equation*}
\begin{split}
& \vec w = \frac{\vec q_1 \cdot \vec q_2}{t_1 - t_2} \\
& \vec q =  \frac{\vec q_1 \cdot \vec q_2}{t_1 - t_2} t + \vec r
\end{split}\quad\Rightarrow\quad
\begin{split}
& (t= t_1) : \; (t_1 - t_2) \vec q_1 =  (\vec q_1 \cdot \vec q_2) t_1 + \vec r (t_1 - t_2) \\
& (t= t_2) : \; (t_1 - t_2) \vec q_2 =  (\vec q_1 \cdot \vec q_2) t_2 + \vec r (t_1 - t_2) 
\end{split}\quad\Rightarrow\quad
\begin{split}
& \vec q = \Big( \frac{\vec q_1 \cdot \vec q_2}{t_1 - t_2} \Big) t + \frac{\vec q_2 t_1 - \vec q_1 t_2}{t_1 - t_2} \\
& \dot \vec q =  \frac{\vec q_1 \cdot \vec q_2}{t_1 - t_2}  
\end{split}
\end{equation*}
where the last two equations explicit the boundary conditions. 
The Lagrangian of a free motion became
\begin{equation*}
\begin{split}
L = \frac{1}{2} m \dot \vec q^2 = \frac{1}{2} m \Big( \frac{\vec q_1 \cdot \vec q_2}{t_1 - t_2} \Big)^2
\end{split}
\end{equation*}
And the minimal Action is written as follow and represents \say{the true trajectory}
\begin{equation*}
\begin{split}
\int_{t_1}^{t_2} dt L = \frac{1}{2} m \Big(   \frac{\vec q_1 \cdot \vec q_2}{t_1 - t_2}  \Big)^2 (t_1 - t_2)  = S_{min}
\end{split}
\end{equation*}

\begin{center}
\fbox{\begin{minipage}{42em}
\begin{center}
\large
\say{\emph{The real motion is given by the minimum action.}}
\normalsize
\end{center}
\end{minipage}}
\end{center}

\paragraph{Hamiltonian equation}
What happen with the generic lagrangian instead substituting the free motion path?
Starting from the lagrangian equation:
\begin{equation*}
\begin{split}
\frac{d }{d t } \Big( \frac{\partial L}{\partial \dot q_i } \Big) - \frac{\partial L}{\partial q_i }  & = 0  \quad\quad  \Big[ \mbox{where}\quad \dot q_i = p_i \Big]  \\ 
\dot q_i \frac{d }{d t } \Big( \frac{\partial L}{\partial \dot q_i } \Big) - \dot q_i \frac{\partial L}{\partial q_i } & = 0 \\
\frac{d }{d t } \Big( \dot  q_i \frac{\partial L}{\partial \dot q_i } \Big) - \underbrace{ \dot  q_i \frac{\partial L}{\partial \dot  q_i} - \dot  q_i \frac{\partial L}{\partial  q_i } } & = 0 \\
\frac{d }{d t } \Big( \dot  q_i \frac{\partial L}{\partial \dot q_i } \Big) -  \frac{d }{d t } \Big( L (q_i, \dot q_i) \Big) & = 0
\end{split}\quad\Rightarrow\quad
\begin{split}
\frac{d }{d t } \Big( \dot q_i p_i - L(q_i, \dot q_i) \Big) = 0
\end{split}
\end{equation*}

Given the lagrangian as $L(q_i, \dot q_i)$ let's check its \emph{invariance} when the coordinates change: the lagrangian must not change.
For small $\varepsilon$
\begin{equation*}
\begin{split}
q_i  & \to q_i + \varepsilon_i f_i(q_1, \ldots, q_i, \ldots, q_N) \\
\dot q_i  & \to \dot q_i + \varepsilon_i \frac{d }{d t }  f_i(q_1, \ldots, q_i, \ldots, q_N)
\end{split}
\end{equation*}
We can write the variation of the lagrangian as
\begin{equation*}
\begin{split}
\sum \frac{\partial L}{\partial q_i } \delta q_i + \frac{\partial L}{\partial \dot q_i } \delta \dot q_i & = 0 \\
\sum \underbrace{ \frac{\partial L}{\partial q_i } } f_i + \underbrace{ \frac{\partial L}{\partial \dot q_i }} \frac{d }{d t } f_i & = 0 \\
\frac{d }{d t } \Big( \frac{\partial L}{\partial \dot q_i }  \Big) \quad\quad \tau_i \quad\quad\; &
\end{split}\quad\quad\quad\longrightarrow\quad
\begin{split}
\frac{d }{d t } \sum_i p_i f_i = 0
\end{split}
\end{equation*}
Since the lagrangian is \emph{invariant} for transformation, as we just saw, this leads to a \emph{conservation} $\leftrightarrow$ \emph{symmetry}.
This concept is also known as \emph{Noether theorem}.


\vspace{10mm}
\large 
Now ...something here not so clear...

\normalsize
about changing formalism from Classical Mechanic to Quantum Mechanic.
\vspace{10mm}


\paragraph{Hamiltonian and lagrangian density}
Since the beginning of the course we introduce, without saying, and use the hamiltonian density $\mathcal{H}$ instead of the hamiltonian $H$, that is defined as:
\begin{equation*}
H = \int_0^L dx \, \mathcal{H}
\end{equation*}
as well as the lagrangian density $\mathcal{L}$ instead of the lagrangian L:
\begin{equation*}
L = \int_0^L dx \, \mathcal{L} \quad\quad\quad\quad \mbox{( ! controllare se è corretto ! )}
\end{equation*}


\paragraph{Back to the \emph{rod}} using now in hamiltonian formalism
\begin{equation*}
\begin{split}
\mathcal{H} & = \frac{1}{2} \dot \phi^2 + \frac{1}{2} v^2 (\partial_x \phi) 
= \frac{1}{2} p^2 + \frac{1}{2} v^2 (\partial_x \phi) \\
\mathcal{L}  = p \, \dot q - \mathcal{H} & = \dot\phi \dot\phi - \Big( \frac{1}{2} \phi^2  +  \frac{1}{2} v^2 (\partial_x \phi)^2  \Big) 
= \frac{1}{2} \dot \phi^2 - \frac{1}{2} v^2 (\partial_x \phi)^2 
\end{split}\quad\quad\quad\quad
\begin{split}
\mbox{where}\quad p = \dot \phi 
\end{split}
\end{equation*}
when the velocity is set as $v = \Omega a$ it depends strictly on the material of the medium.
So when moving in the vacuum, and not along a rod, what happens? 
The velocity will be set to $v = c = 1$, as natural units are used, and \say{forget the mass} such that $m=1$.
The lagrangian density became consistent with this
\begin{equation}
\begin{split}
\mathcal{L} = \frac{1}{2} \dot \phi^2 - \frac{1}{2} v^2 (\partial_x \phi)^2 
\end{split}
\label{eq:lagrangian_phi}
\end{equation}
The definition of \emph{field} goes with the definition of a proper \emph{energy density}.
The action became:
\begin{equation*}
\begin{split}
S = \int_{t_1}^{t_2} dt \, L =  \int_{t_1}^{t_2} dt \int_0^L dx \, \mathcal{L}
\end{split}
\end{equation*}




\newpage
\section{7 ottobre}
Continuing from the lagrangian \ref{eq:lagrangian_phi} 
\begin{equation*}
\begin{split}
\mathcal{L}  = \frac{1}{2} \dot \phi^2 - \frac{1}{2} v^2 (\partial_x \phi)^2 = \frac{1}{2} (\partial_t \phi)^{2}  - \frac{1}{2} v^2 (\partial_x \phi)^{2}
\end{split}
\end{equation*}
we see the lagrangian as function of these variables $\mathcal{L} = \mathcal{L}(\phi, \, \partial_t \phi, \, \partial_x \phi)$ and we write the variation and the variational principle as follow
\begin{equation*}
\begin{split}
S =  \int \dd t \int \dd x \, \mathcal{L} =  \int \dd t L
\end{split}\quad\quad\quad\quad\quad
\begin{split}
\var{S} = 0
\end{split}
\end{equation*}
\begin{equation*}
\begin{split}
\var{S} & = \int \dd t \int \dd x \, \mathcal{L}(\phi + \var{\phi}, \dot \phi + \pdv{t} \var{\phi}, \partial_x \phi + \partial_x \var{\phi}) 
- \int \dd t \int \dd x \, \mathcal{L}(\phi , \, \dot \phi , \, \partial_x \phi ) \\
& = \int\dd t\int \dd x\, \qty\bigg{\pdv{\mathcal{L}}{\phi} \var{\phi} - \pdv{\mathcal{L}}{(\partial_t \,\phi)} \partial_t \var{\phi} - \pdv{\mathcal{L}}{(\partial_x \,\phi) } \partial_x \var{\phi}} \\
& = \int \dd t \int \dd x \, \qty\bigg{  \pdv{\mathcal{L}}{\phi} - \partial_t \pdv{\mathcal{L}}{(\partial_t \,\phi)} - \partial_x \pdv{\mathcal{L}}{(\partial_x \,\phi)}  } \var{\phi} 
\end{split}
\end{equation*}
Hence we may set some boundary conditions at times $t_1$ and $t_2$, for the rod problem
\begin{equation*}
\begin{split}
& \var{\phi}(\vec x, t_1) = 0 \\
& \var{\phi}(\vec x, t_2) = 0
\end{split}
\end{equation*}
Notice that the partial derivative of $\mathcal{L}$ with respect to $\phi$ is the following, 
and trough which we find the \emph{Euler-Lagrange equation} in \emph{Quantum Field Theory}
\begin{equation*}
\begin{split}
\pdv{\mathcal{L}}{\phi} & = \partial_t \pdv{\mathcal{L}}{(\partial_t \phi)} + \partial_x \pdv{\mathcal{L}}{(\partial_x \phi)}  
+\partial_y \pdv{\mathcal{L}}{(\partial_y \phi)} +\partial_z \pdv{\mathcal{L}}{(\partial_z \phi)} \\
\mbox{with} \quad \partial_{\mu} & = (\partial_1, \partial_2, \partial_3, \partial_4) = (\partial_t, \partial_x, \partial_y, \partial_z)
\end{split}
\quad\quad\Rightarrow\quad\quad
\begin{split}
\partial_{\mu} \pdv{\mathcal{L}}{(\partial_{\mu} \phi)} = \pdv{\mathcal{L}}{\phi}
\end{split}
\end{equation*}
The lagrangian density may be generalized in 3-D:
\begin{equation*}
\begin{split}
\mathcal{L} = & \frac{1}{2} (\partial_t \phi)^2 - \frac{1}{2} (\partial_x \phi)^2 - \frac{1}{2} (\partial_y \phi)^2 - \frac{1}{2} (\partial_z \phi)^2 \\
= &  \frac{1}{2} (\partial_{\mu} \phi)^2
\end{split}
\quad\quad\quad\quad\quad \mbox{where} \quad (\partial_{\mu} \phi)^2  = \partial_{\mu} \phi \,\partial_{\mu} \phi
\end{equation*}
This shorter notation implicitly implement the relativity, with $v=1$ and natural units.
If you want to describe a physical system, like the rod, you must instead keep it explicit with $v \neq 0$ 
and the Lagrangian will need one term more, such that
\begin{equation*}
\mathcal{L} = - \frac{1}{2} (\partial_{\mu} \phi)^2 - \frac{1}{2} m^2 \phi^2
\end{equation*}

\paragraph{Solve the motion equation} starting from the lagrangian above, we derive it w.r.t. $\phi$ and w.r.t. $\partial_{\mu}\phi$
\begin{equation*}
\begin{split}
& \mathcal{L} = - \frac{1}{2} (\partial_{\mu} \phi)^2 - \frac{1}{2} m^2 \phi^2 \\
&\begin{cases}
\quad \pdv{\mathcal{L}}{\phi} & = - m^2 \phi \\
\pdv{\mathcal{L}}{(\partial_{\mu}\phi)} & = - (\partial_{\mu}\phi)
\end{cases}
\end{split}
\end{equation*}
Using the \emph{laplacian} we find 
\begin{equation*}
\begin{split}
\partial_{\mu}\,\partial_{\mu} & = \partial_{1}\,\partial_{1} + \partial_{2}\,\partial_{2} + \partial_{3}\,\partial_{3} - \partial_{t}\,\partial_{t} \\
& =  \nabla_{1}\,\nabla_{1} + \nabla_{2}\,\nabla_{2} + \nabla_{3}\,\nabla_{3} - \nabla_{t}\,\nabla_{t} \\
& = \nabla^{2} - \partial_{t}^{2} \, = \Box  
\end{split}
\quad\quad\Rightarrow\quad
\begin{split}
\partial_{\mu} (- \partial_{\mu} \phi ) & = - m^2 \phi \\
- \Box \, \phi  & = - m^2 \phi
\end{split}
\quad\quad\Rightarrow\quad ( \Box - m^2 ) \, \phi  = 0
\end{equation*}
obtaining the \emph{Klein-Gordon equation}.

\begin{equation*}
\begin{split}
\phi = e^{ i \, ( \vec p \cdot \vec x - \phi_0 \, t  ) } = e^{ i \, P_{\mu} x_{\mu} } 
\quad\Rightarrow\quad 
\Big(  -\vec P^2 + P_0^2 - m^2  \Big) = 0
\quad\Rightarrow\quad
P_0^2  = \vec P^2 + m^2  
\end{split}
\end{equation*}
that describes the relativistic motion of a particle of mass $m$.
Then the \emph{free particle} may have a positive or negative energy value $P_0$
\begin{equation*}
P_0 = \pm \sqrt{ \vec P^2 + m^2  }
\end{equation*}

\paragraph{Time evolution} If we also want to consider the time dependency as QM time evolution we may write 
\begin{equation*}
\begin{split}
e^{- i \,E \, t} = e^{- i \, (m + K_E)} 
\end{split}\quad\Rightarrow\quad
\begin{split}
\phi = e^{- i \, m \, t} \psi
\end{split}\quad\Rightarrow\quad
\begin{split}
\partial_t \phi = \dot \phi = \Big(   (-i\,m) e^{- i \, m \, t} \psi + e^{- i \, m \, t} \dot \psi   \Big) 
\end{split}
\end{equation*}
and also the second time derivative
\begin{equation*}
\begin{split}
\partial^2_t \phi =  \ddot \phi = &  \Big(   (-i\,m)  (-i\,m) e^{- i \, m \, t} \psi +  (-i\,m) e^{- i \, m \, t} \dot \psi +  (-i\,m) e^{- i \, m \, t} \dot \psi 
+ \underbrace{e^{- i \, m \, t} \ddot \psi }  \Big) \\
= & \Big(  -m^2 e^{- i \, m \, t} \psi  - 2 i\,m e^{- i \, m \, t} \dot \psi   \Big) 
\end{split}
\end{equation*}
With the approximation to the first order derivative $ e^{- i \, m \, t} \ddot \psi = 0 $ .
Finally we find the solution that's something familiar to us
\begin{equation*}
\begin{split}
& e^{- i \, m \, t} \Big(  \nabla^2 \psi + 2 i\,m \dot \psi + m^2 \psi - m^2 \psi  \Big) = 0 \\
& \quad\Rightarrow\quad i \dot \psi  = - \frac{i}{2m} \nabla^2 \psi
\end{split}
\quad\quad \mbox{where} \quad \nabla^2 \phi = e^{- i \, m \, t} \nabla^2 \psi 
\end{equation*}
that's the \emph{Schrodinger equation} with $\hbar = 1$ that is the non-relativistic limit of the \emph{Klein-Gordon equation} that is a relativistic equation.



\newpage
\section{11 ottobre}
The electromagnetic field theory approach starts with a postulate: \emph{it is possible to derive the Maxwell's equations starting from a lagrangian}.
From this postulate, we now ask to ourself which is that lagrangian that solves the Euler-Lagrangian equation (\ref{eq:EulerLagranEq})?
\begin{equation*}
\begin{split}
\partial_{\mu} \pdv{\mathcal{L}}{(\partial_{\mu} \phi)} = \pdv{\mathcal{L}}{\phi}
\end{split} 
\quad\quad\quad\mbox{where}\quad
\begin{split}
\phi = 
\begin{cases}
\phi \quad\quad \mbox{scalar potential} \\
\vec A \quad\quad \mbox{vector potential} 
\end{cases}
\end{split}
\label{eq:EulerLagranEq}
\end{equation*}
To do so, we start postulating that it may be
\begin{equation*}
\begin{split}
\mathcal{L} = \frac{1}{2} \Big( \vec E^2 - \vec B^2 \Big) - \rho \, \phi + \vec J \cdot \vec A
\end{split}
\quad\quad\quad\mbox{where}\quad
\begin{split}
\begin{cases}
\phi \, & \mapsto \quad \vec E = - \vec \nabla \phi - \pdv{\vec A}{t}  \\
\vec A \, & \mapsto \quad \vec B = \vec \nabla \times \vec A 
\end{cases}
\end{split}
\end{equation*}
Where the magnetic field's components may be also written as $\, B_i = \epsilon_{i j k} \partial_j \, A_k$, using the Levi-Civita symbol.
We can define a 4-vector that transforms following the Lorentz transformations as $\, A_{\mu} = (\vec A , i \, \phi) $.

\paragraph{Maxwell's equations regression}
\begin{equation*}
\begin{split}
\begin{cases}
\vec \nabla \cdot \vec B = 0 \\
\vec \nabla \times \vec B - \pdv{\vec E}{t} = \vec J   \quad\quad \mbox{corrent density - \emph{source}} \\
\vec \nabla \cdot \vec E = \rho   \quad\quad\quad\quad\quad \; \mbox{charge density - \emph{source}} \\
\vec \nabla \times \vec E + \pdv{\vec B}{t} = 0
\end{cases}
\end{split} 
\Big\} \Rightarrow\quad \mbox{continuity equation} \quad
\begin{split}
\pdv{ \rho}{t} = - \vec \nabla \cdot \vec J 
\end{split}
\end{equation*}
Now, similarly as before, we define a 4-vector that transforms following the Lorentz transformations as $\, J_{\mu} = (\vec J , i \, \rho) $.
Which leads to the following lagrangian
\begin{equation*}
\begin{split}
\mathcal{L} = \frac{1}{2} \Big( \vec E^2 - \vec B^2 \Big) - \rho \, \phi + J_{\mu}\,A_{\mu}
\end{split}
\end{equation*}
What's left is to check if our new postulates and definitions are compatible with what we already know.
\begin{equation*}
\begin{split}
\Big( \ldots \,calculations\, \ldots \Big)
\end{split}
\end{equation*}

\paragraph{Electromagnetic tensor} One we've our new lagrangian we may now introduce the \emph{electromagnetic tensor} $F_{\mu\nu}$ , as follow:
\begin{equation*}
\begin{split}
F_{\mu\nu} = 
\mqty[
0 	& B_3 	& - B_2 	& - i\,E_1 \\
- B_3 & 0     	&  B_1   	& - i\,E_2 \\
B_2 	& - B_1   	&  0   	& - i\,E_3 \\
i\,E_1	& i\,E_2	& i\,E_3 	& 0
]
\end{split}
\quad\Rightarrow\quad
\mbox{\emph{antisimetrical tensor}}
\label{eq:electromagneticTensor}
\end{equation*}
and also 
\begin{equation*}
\begin{split}
F_{\mu\nu} = 
\partial_{\mu} A_{\nu} - \partial_{\nu} A_{\mu}
\end{split}\quad \Longleftrightarrow \quad
\begin{split}
\begin{cases}
B_i = & \epsilon_{ijk} \, \partial_j \, A_k = \Big( \vec \nabla \times \vec A \Big)_i \\
B_i = & \frac{1}{2} \epsilon_{ijk} \, F_{jk} \quad \mbox{\small(inverse relation)} \\
E_i = & F_{i\,4} = - F_{4 \, i} 
\end{cases}
\end{split}
\end{equation*}

\begin{equation*}
\begin{split}
examples:
\end{split}
\begin{split}\quad
\begin{cases}
F_{1\,4} = \partial_{1} A_{4} - \partial_{4} A_{1} = i\,\partial_{1} \phi - i\, \pdv{A_1}{t} = i\, (-E_1 ) = - i\, E_1 \\
F_{1\,2} = \partial_{1} A_{2} - \partial_{2} A_{1} = \Big( \vec \nabla \times \vec A \Big)_3 =
\mdet{\partial_{1} & \partial_{2} \\ A_1 & A_2}_{\hat{k}} =   (\partial_{1} A_{2} - \partial_{2} A_{1})_{\hat{k}} = B_3
\end{cases}
\end{split}
\end{equation*}
This leads to a compact way to write the Maxwell's equations, in only one tensorial equation:
\begin{equation*}
\begin{split}
\begin{cases}
\partial_{\mu} F_{i \, \mu} = J_{i} \\
\partial_{i} F_{4 \, i} = J_{4}\\
\end{cases}
\end{split} \quad\Rightarrow\quad
\partial_{\mu} F_{\nu \mu} = J_{\nu}  \; .
\end{equation*}


\paragraph{The electromagnetic tensor and its dual}
Starting from the definition \ref{eq:electromagneticTensor} of the electromagnetic tensor, we may see
\begin{equation*}
\begin{split}
& F_{\mu \nu} F_{\mu \nu} = - F_{\mu \nu} F_{\nu \mu} = - T_{\mu} (F \cdot F) = 2 (\vec B^2 - \vec E^2) \\
& \mbox{from which} \quad \mathcal{L} = - \frac{1}{4} F_{\mu \nu} F_{\mu \nu} + J_{\mu} A_{\mu}
\end{split}
\end{equation*}
Trough the \emph{antisymmetric tensor} $T_{\mu \nu \lambda}$ it's possible to write the following identity:
\begin{equation*}
\partial_{\mu} F_{\nu \lambda} + \partial_{\nu} F_{\lambda \mu} + \partial_{\lambda} F_{\mu \nu} = 0
\end{equation*}

The \emph{dual tensor} of $F$ is as well invariant for \emph{cage transformations}
\begin{equation*}
\begin{split}
\tilde{F}_{\mu \nu} \equiv \frac{1}{2} \epsilon_{\mu \nu \lambda \rho} F_{\lambda \rho} 
= \mqty[
0 	  & - i\,E_3 	&  i\,E_2 	& B_1 \\
i\,E_3 & 0     	&  - i\,E_1   	& B_2 \\
-i\,E_2 & i\,E_1   	&  0   	& B_3 \\
-B_1	  & -B_2	& -B_3 	& 0
]
\end{split} \quad\quad
\begin{split}
\quad\quad & \mbox{Levi-Civita symbols:} \\
&\quad \mbox{3-D} \quad \epsilon_{\mu \nu \lambda}  \;\,\quad\rightarrow\quad \epsilon_{1\, 2\, 3} = 1 \\
&\quad \mbox{4-D} \quad \epsilon_{\mu \nu \lambda \rho}  \quad\rightarrow\quad \epsilon_{1\, 2\, 3\, 4} = - 1
\end{split}
\end{equation*}
$\quad\leadsto \,$ An interesting \underline{exercise}, calculate: $\,F_{\mu \nu} \tilde{F}_{\mu \nu} = - 4 i \, \vec E \cdot \vec B \;$ and verify the equivalence.

\begin{center}
\fbox{\begin{minipage}{42em}
\begin{center}
\large
Confusione con gli indici?
\normalsize
Il Prof. considera, al fine del corso, equivalente scrivere:
\begin{equation*}
A_{\mu} B^{\mu}  \; \longleftrightarrow \;  A_{\mu} B_{\mu} 
\end{equation*}
quindi attenzione, soprattutto rispetto alla notazione usata nel corso di \emph{General Relativity}.
\end{center}
\end{minipage}}
\end{center}

We know want to demonstrate that the tensor $F$ is \emph{cage invariant}, 
so let's introduce the scalar quantity $\Lambda$ and write the transformation:
\begin{equation*}
\begin{split}
A_{\mu} \;\longrightarrow\; A_{\mu} + \partial_{\mu} \Lambda
\end{split}
\end{equation*}
\begin{equation*}
\begin{split}
F_{\mu \nu} = \partial_{\mu} A_{\nu} - \partial_{\nu} A_{\mu}  \;\longrightarrow\; 
F_{\mu \nu} & =  \partial_{\mu} A_{\nu} + \partial_{\mu}\partial_{\nu} \Lambda - \partial_{\nu} A_{\mu} - \partial_{\nu}\partial_{\mu} \Lambda \\
& = \partial_{\mu} A_{\nu} - \partial_{\nu} A_{\mu} 
\end{split}
\end{equation*}
We may say that $F_{\mu \nu}$ \say{has the right combination of the magnetic and electron field}, to make two terms cancelled each other. \\
A generic vector changes under Lorentz transformations as
\begin{equation*}
C_{\mu} \;\longrightarrow\; L_{\mu \nu} C_{\nu}
\end{equation*}
How does the $F_{\mu \nu}$ tensor changes under Lorentz transformations?
\begin{equation*}
F_{\mu \nu}  \;\longrightarrow\;   L_{\mu \mu'} L_{\nu \nu'}  F_{\mu' \nu'}
\end{equation*}

\newpage
\section{13 ottobre}


\newpage
\section{14 ottobre}


\newpage
\section{18 ottobre}


\newpage
\section{20 ottobre}


\newpage
\section{21 ottobre}


\newpage
\section{25 ottobre}


\newpage
\section{27 ottobre}






\newpage
\section{28 ottobre}

















\end{document}
