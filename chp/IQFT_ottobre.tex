%%
%% Author: Dario Chinelli
%% begin 2022-10-16
%% last mod 2022-12-24
%%


%%  NON ESEGUIRE QUESTO FILE !


% Preamble
\documentclass[class=article]{standalone}

% Packages
\usepackage[subpreambles=true]{standalone}
\usepackage{import}
\usepackage{graphicx}
\usepackage{amsmath}


% Document
\begin{document}

\section{The rod}

Given a 1-dimensional rod composed by N-particles, linked each others with a "spring", the hamiltonian density is

\begin{equation*}
\begin{split}
\mathcal{H} = \frac{1}{2} \sum_{n=1}^{N} \Big[  P_n^2 + \Omega^2 (q_n - q_{n+1})^2 + \Omega_0^2 q_n^2 \Big]
\end{split}
\end{equation*}

where the last term $\Omega_0^2 q_n^2$ is relative to the equilibrium position of the n-particle.
The \emph{periodic boundaries conditions} to $N \to \infty$ and $a \to 0$.


On the other side we can write the Newtonian equation as 

\begin{equation*}
\begin{split}
& H = \frac{1}{2} \int_{0}^{L} dx \Big[ p^2(x) + v^2 \Big( \frac{\partial q(x)}{\partial x} \Big) \Big] \\
% & \mbox{where} \\
&  p(x) = \dot q(x) \\
&  \ddot q(x) = v^2 \frac{\partial^2 q(x)}{\partial x^2}
\end{split}
\end{equation*}

the solution inside the boundaries is
\begin{equation*}
\ddot q_n = \Omega^2 \Big( q_{n+1} + q_{n-1} -2 q_n \Big)
\end{equation*}


\paragraph{Normal modes} or normal frequencies
\begin{equation*}
\begin{split}
& q_n = \sum_j e^{i j n} \frac{Q_j}{\sqrt{N}} \\
& q(x) =  \frac{1}{\sqrt{a} } \sum_n e^{ \frac{2 \pi l}{N a} (n a) } \frac{Q_j}{\sqrt{N}}  = \frac{1}{\sqrt{a} }  \sum_k e^{i k x} \frac{Q_k}{\sqrt{N}} \\
& \quad\quad k = \frac{2 \pi l}{L} \\
& \Rightarrow\quad q(x) =  \sum_k e^{i k x} \frac{Q_k}{\sqrt{N a}} =  \sum_k e^{i k x} \frac{Q_k}{\sqrt{L}}
\end{split}
\end{equation*}

Considering now the Newtonian equation, $p^2(x) = \dot q^2(x) $, 
$ \sum_{n=1}^N e^{i n (j-j')} = \delta_{j, j'}$ where $j = \frac{2\pi l}{N}$, 
we can move from the sum to the integral using the following relation 
$ \sum_{n=1}^N \to \frac{1}{a} \int_{0}^{L} dx$ and this leads to
$\int_{0}^{L} dx e^{i (k-k') x} = L \delta_{k, k'}$ .

Somehow we may land on this following expression:
\begin{equation*}
\frac{1}{L} \sum_{k, k'} L \delta_{k, k'} Q_k \dot Q_{k'} = \sum_k Q_k \dot Q_{k} =   \sum_k | \dot Q_{k} |^2
\end{equation*}
To finally get a total classical description, a discrete sum on a numerable set, as follow

\begin{equation*}
H = \frac{1}{2}  \sum_k  | \dot Q_{k} |^2 + k^2 v^2  | Q_{k} |^2
\end{equation*}








\end{document}
